\documentclass[a4paper, conference]{IEEEtran}
\IEEEoverridecommandlockouts
% The preceding line is only needed to identify funding in the first footnote. If that is unneeded, please comment it out.
%Template version as of 6/27/2024

\usepackage{cite}
\usepackage{amsmath,amssymb,amsfonts}
\usepackage{algorithmic}
\usepackage{graphicx}
\usepackage{textcomp}
\usepackage{xcolor}
\usepackage{listings}
\usepackage{graphicx}
\usepackage[ 
   colorlinks,        % Links ohne Umrandungen in zu wählender Farbe 
   linkcolor=black,   % Farbe interner Verweise 
   filecolor=black,   % Farbe externer Verweise 
   citecolor=black,   % Farbe von Zitaten 
   urlcolor=blue	  % Farbe von Links
   ]{hyperref} %Verlinkungen
\lstdefinestyle{yaml}{
    basicstyle=\color{blue}\footnotesize,
    rulecolor=\color{black},
    string=[s]{'}{'},
    stringstyle=\color{blue},
    comment=[l]{:},
    commentstyle=\color{black},
    morecomment=[l]{-}
}

\def\BibTeX{{\rm B\kern-.05em{\sc i\kern-.025em b}\kern-.08em
    T\kern-.1667em\lower.7ex\hbox{E}\kern-.125emX}
}

\usepackage[ngerman]{babel} % Anführungszeichen Björn

\newcommand{\quotationMarkGerman}[1]{\glqq{}#1\grqq{}}

\newcommand{\topic}{Zwischenbericht: Pathogensuche mit NGS}
\newcommand{\authorA}{Björn Emanuel Fürtges}
\newcommand{\indexTerms}{mNGS, Pathogen Detection, Nachweis von Krankheitserregern}




\hypersetup{%
  pdftitle={\topic},
  %pdfsubject={\subtopic},
  pdfauthor={\authorA},
  pdfkeywords={HTW, Master Angewandte Informatik, Wintersemester 2024/25, \indexTerms}
}  

\begin{document}

\title{\topic
}

\author{\IEEEauthorblockN{\authorA}
\IEEEauthorblockA{\textit{FB4 Informatik, Kommunikation und Wirtschaft: Master Angewandte Informatik} \\
\textit{Hochschule für Technik und Wirtschaft Berlin}\\
Berlin, Deutschland \\
Bjoern.Fuertges@Student.HTW-Berlin.de}
}

\maketitle

\begin{abstract}
Dieser Zwischenbericht gibt einen kurzen Zwischenstand zur Seminararbeit zum Thema \quotationMarkGerman{Pathogensuche mit NGS}. Metagenomisches Next-Generation-Sequencing (mNGS) kann zur Suche nach Krankheitserregern verwendet werden und verspricht dabei zuverlässigere Ergebnisse als \quotationMarkGerman{klassische} Methoden.
\end{abstract}

\begin{IEEEkeywords}
\indexTerms
\end{IEEEkeywords}

\section{Einleitung}
Metagenomisches Next-Generation-Sequencing (mNGS) ermöglicht die Diagnose von Infektionen des zentralen Nervensystems (ZNS) \cite{clinicalMetagenomicNextGenerationSequencing}. Mithilfe dieser Methode können sowohl seltene als auch bislang unbekannte Pathogene identifiziert werden \cite{clinicalMetagenomicNextGenerationSequencing}.

\section{Vorteile und derzeitige Limitierungen}
\subsection{Vorteile}
mNGS ermöglicht die Identifikation sowohl neuer als auch bekannter Pathogene, insbesondere in Fällen, in denen andere diagnostische Verfahren scheitern \cite{clinicalMetagenomicNextGenerationSequencing}. Darüber hinaus liefert das Verfahren genetische Informationen über die Pathogene, die für weitere Forschungen genutzt werden können \cite{clinicalMetagenomicNextGenerationSequencing}. Daher wird mNGS vor allem eingesetzt, wenn \quotationMarkGerman{klassische} Methoden keine oder nur ungenaue Ergebnisse liefern \cite{weiGuSteveMillerCharlesChiuMNGSPathogenDetection}.

\subsection{Derzeitige Limitierungen}
Kontaminationen, unzureichende Datenbanken und die Komplexität der Datenanalyse können die Auswertung der Befunde anfällig für Fehlinterpretationen machen \cite{clinicalMetagenomicNextGenerationSequencing, weiGuSteveMillerCharlesChiuMNGSPathogenDetection}. Darüber hinaus sind die Kosten für das Verfahren sehr hoch \cite{clinicalMetagenomicNextGenerationSequencing, weiGuSteveMillerCharlesChiuMNGSPathogenDetection}. Gleichzeitig ist die Bearbeitungszeit mit 1--3 Wochen relativ lang \cite{clinicalMetagenomicNextGenerationSequencing}, und die Verfügbarkeit des Verfahrens ist eingeschränkt \cite{clinicalMetagenomicNextGenerationSequencing}. Ein weiterer kritischer Punkt sind die Referenzdatenbanken: Sind diese unvollständig, können Pathogene möglicherweise nicht erkannt werden oder ihre Identifikation wird erschwert.

\section{Derzeitige klinische Bedeutung}
In den Arbeiten von Su et~al. \cite{clinicalMetagenomicNextGenerationSequencing} und Gu et~al. \cite{weiGuSteveMillerCharlesChiuMNGSPathogenDetection} wird beschrieben, dass mNGS derzeit meist als letzte diagnostische Möglichkeit betrachtet und eingesetzt wird. Es sind jedoch weitere klinische Studien erforderlich, um die bisherigen Ergebnisse zu validieren.

\section{Identifikation von möglichen Themen für die Seminararbeit}
Im Folgenden werden Themengebiete genannt, die für eine Seminararbeit in Betracht gezogen werden könnten. Diese Aufzählung ist nicht abschließend.

\paragraph{Datenverarbeitung}
Im Rahmen der Seminararbeit könnten Methoden zur Verarbeitung der mNGS-Daten untersucht werden. Dabei liegt der Fokus auf der Analyse der Daten und den verschiedenen Verfahren zur Klassifikation von Pathogenen.

\paragraph{Referenzdatenbanken}
Ein weiterer möglicher Schwerpunkt wäre die Untersuchung von Ansätzen zur Verbesserung von Referenzdatenbanken, da diese eine entscheidende Limitierung des Verfahrens darstellen.

\section{Ausblick und Fazit}
Erste Literaturrecherchen zeigen, dass mNGS viele Vorteile bietet, jedoch auch derzeitigen Einschränkungen unterliegt. Zudem fehlen umfassende klinische Studien, um die Methode besser zu validieren. Abschließend wurden zwei Themengebiete vorgestellt, die für eine Vertiefung in der Seminararbeit relevant sein könnten. Diese werden in zukünftigen Besprechungen weiter diskutiert und konkretisiert.   

\bibliography{literatur}

\bibliographystyle{IEEEtran}

\end{document}
